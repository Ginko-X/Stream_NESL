\documentclass[a4paper]{article}

\usepackage{proof}
\usepackage{amsmath, amssymb, amsthm}
\usepackage{mathtools}
\usepackage{bussproofs}
%\usepackage{color}
\usepackage{geometry}
\usepackage{enumerate}
\usepackage{extarrows}
\usepackage{verbatim}
 
\geometry{left=2.5cm,right=2.5cm,top=3cm,bottom=3cm}

\newtheorem{thm}[equation]{Theorem}
\newtheorem{lem}[equation]{Lemma}
\newtheorem{prop}[equation]{Proposition}
\newtheorem{cor}[equation]{Corollary}
\newtheorem{conj}[equation]{Conjecture}

\def\*#1{\mathbf{#1}}

%source language syntax
\def\Let#1#2#3{\*{let} \ #1 = #2 \ \*{in} \ #3}
\def\Comp#1#2#3#4{\{#1 : #2 \ \*{in} \ #3 \ \*{using} \ #4\}}

\def\hcall{\phi} 

\def\const#1{\*{const}_{#1} ()}
\def\iota#1{\*{iota} (#1)} 
\def\plus#1#2{\*{plus} (#1,#2)}



%types
\def\int{\*{int}}
\def\tseq#1{\{#1\}}  


% symbol shorthands
\def\Eva{\downarrow}
\def\Ra{\Rightarrow}
\def\Env{\ \vdash\ } 

\def\Seqk#1{\{ #1_1,...,#1_k \}}
\def\Tupk#1{(#1_1,...,#1_k)}


% greek letters
\def\sgm{\sigma}
\def\Gam{\Gamma}
\def\del{\delta}


% special font
\def\MC{\mathcal{C}} 
\def\ME{\mathcal{E}} 
\def\MP{\mathcal{P}} 
\def\MT{\mathcal{T}} 
\def\MV{\mathcal{V}}



% Judgment
\def\Jug#1{Judgment \boxed{#1} \\[1ex]}

\def\Map#1#2{#1 \mapsto #2}
 

% Evaluation
\def\Eval#1#2#3{#1 \Env #2 \Eva #3 } 
\def\EvalF#1{Eva_\hcall#1}  % function evaluation

% target semantics 
\def\TaSem#1#2#3{\langle #1,#2 \rangle \Eva #3}  
\def\Talet#1#2{\mathtt{let} \ #1 \ \mathtt{in} \ #2}

% compiling/translation
\def\Trans#1#2#3#4#5{#1 \Env #2 \xRightarrow[#4]{#3} #5} 
\def\TransP#1#2#3#4#5{\mathbf{transPlus} (#1,#2) \Ra^{#3}_{#4} #5}

%typing
\def\Type#1#2#3{#1 \Env #2 : #3 } 
\def\TypeF#1{Typ_\hcall#1} 


%value representation
\def\ValRep#1#2#3{ #1 \triangleright_{#2} #3}


% rule with a name
\def\ERule#1#2{\textsc{E-#1} : \vcenter{#2}}
\def\PRule#1#2{\textsc{P-#1} : \vcenter{#2}}
\def\CRule#1#2{\textsc{C-#1} : \vcenter{#2}}

% only rule name
\def\EName#1{\textsc{E-#1}}
\def\PName#1{\textsc{P-#1}}
\def\CName#1{\textsc{C-#1}}

%proof tree shorthands
\def\AC#1{\AxiomC{$#1$}}
\def\Axiom#1{\AxiomC{} \UnaryInfC{$#1$}}
\def\UC#1{\UnaryInfC{$#1$}}
\def\BC#1{\BinaryInfC{$#1$}}
\def\TC#1{\TrinaryInfC{$#1$}}
\def\DP{\DisplayProof}
\def\LeLa#1{\LeftLabel{$#1$}}
\def\RiLa#1{\RightLabel{$#1$}}
\def\PT#1{#1 \DisplayProof \qquad} 
%UnaryInfC with a deriviation name
\def\UCN#1#2{\AxiomC{#1} \noLine \UC{#2}}


%color
%\def\Blue{\color{blue}}

%notation
\def\ConEq#1{\xlongequal{<#1}}  % Sgm conditional equation

\def\SetLe{\lessdot} % forall r in set . r < r0  



\title{SNESL formalization}
\author{Dandan Xue}

\begin{document}
\maketitle


\setcounter{section}{-1}

\def\s{\mathit{s}}  
\def\st{\mathit{st}} 

%\def\v{\vec v}
\def\v{\rightharpoonup v}

%\def\vec#1{\overrightarrow{#1}}
\def\unit{()}

\def\vrange#1#2{\langle #1,...,#2 \rangle}

\def\emptyv{\langle \rangle}

\def\phi{\varphi}

\def\ctrl{\mathtt{Ctrl}}
\def\consta#1{\mathtt{Const_{#1}}} 
\def\toflag{\mathtt{ToFlags}}
\def\usum{\mathtt{Usum}}
\def\maptwo{\mathtt{MapTwo}}
\def\scan{\mathtt{ScanPlus}}
\def\sdef#1#2{#1 := #2}
\def\withctrl#1#2#3{#3 := \mathtt{WithCtrl}(#1,#2)}

\def\T{\mathtt{T}} 
\def\F{\mathtt{F}} 
\def\oT{\singl{\T}}
\def\oF{\singl{\F}}

\def\tail#1{\mathtt{tail}(#1)} 

\def\appop{\operatorname{++}}
\def\vapp#1#2{#1 \appop #2}

\def\seval#1#2#3#4{\langle #1,#2 \rangle \Eva^{#3} #4} 
\def\sevalf#1#2#3#4{SEva_{\phi}(#1,...,#2) \Eva^{#3} #4} 
\def\block#1#2#3{Block_{\phi}(#1,...,#2) \Downarrow #3}  
\def\unary#1#2#3{Unary_{\phi}(#1,...,#2) \Downarrow #3}
\def\da{\Downarrow}

\def\blockv#1#2#3#4{Block_{\phi,#4}(#1,...,#2) \Downarrow #3}  
\def\unaryv#1#2#3#4#5{Unary_{\phi,#4}(#1,...,#2) \Downarrow^{#5} #3}



\def\singl#1{\langle #1 \rangle}
\def\Maps#1#2{#1 \rightarrowtail #2}
\def\xR#1#2{\xRightarrow[#1]{#2}}


%translation
\def\sfun#1#2{(#1,#2)}
\def\tranf#1#2#3#4#5{Trans_\phi(#1,...,#2) \xRightarrow[#4]{#3} #5}  

\providecommand{\versionnumber}{0.0.1}

\section{Level-0}
Draft version \versionnumber

\subsection{Source language syntax}
(Ignore empty sequence for now)\\
Expressions:
$$e ::= x \ | \ \Let{x}{e_1}{e_2} \ | \ \hcall{\Tupk{x}} \ | \ \Comp{e}{x}{y}{\cdot} $$
$$\hcall = \*{const}_n \ | \ \*{iota} \ | \ \*{plus} $$ 

Values: \\
$$ n \in \*{Z} $$
$$ v::= n \ | \ \Seqk{v}$$

\subsection{Type system}
$$\tau ::= \int | \tseq{\tau_1}$$

Type environment $\Gamma = [\Map{x_1}{\tau_1}, ..., \Map{x_i}{\tau_i} ]$.
\begin{itemize}

\item \Jug{\Type{\Gam}{e}{\tau}}

\PT{
	\AxiomC{}
	\RiLa{(\Gam(x) = \tau)}
    \UC{\Type{\Gam}{x}{\tau}}
}
\PT{
	\AC{\Type{\Gam}{e_1}{\tau_1}}
	\AC{\Type{\Gam[\Map{x}{\tau_1}]}{e_2}{\tau}}
	\BC{\Type{\Gam}{\Let{x}{e_1}{e_2}}{\tau}}
}\\[1ex]


\PT{
	\AC{\Type{}{\TypeF{\Tupk{\tau}}}{\tau}}
	\RiLa{((\Gam(x_i)= \tau_i)^k_{i=1})}
	\UC{\Type{\Gam}{\hcall{\Tupk{x}}}{\tau}}
}
\PT{
	\AC{\Type{\Gam[\Map{x}{\tau_1}]}{e}{\tau}}
	\RiLa{(\Gam(y)=\tseq{\tau_1})}
	\UC{\Type{\Gam}{\Comp{e}{x}{y}{\cdot}}{\tseq{\tau}}}
}\\[2ex]


\item Auxiliary \Jug{\Type{}{\TypeF\Tupk{\tau}}{\tau}}

\PT{\Axiom{\Type{}{\const{n}}{\int}}}
\PT{\Axiom{\Type{}{\iota{\int}}{\tseq{\int}}}}
\PT{\Axiom{\Type{}{\plus{\int}{\int}}{\int}}}


\end{itemize}

\subsection{Source language semantics}
\begin{itemize}

\item \Jug{\Eval{\rho}{e}{v}}
\PT{
	\AxiomC{}
	\RiLa{(\rho(x)=v)}
	\UC{\Eval{\rho}{x}{v}}
}
\PT{
	\AC{\Eval{\rho}{e_1}{v_1}}
	\AC{\Eval{\rho[\Map{x}{v_1}]}{e_2}{v}}
	\BC{\Eval{\rho}{\Let{e_1}{x}{e_2}}{v}}	
}\\[1ex]

\PT{
	\AC{\Eval{}{\EvalF{\Tupk{v}}}{v}}
	\RiLa{((\rho(x_i)=v_i)^k_{i=1})}
	\UC{\Eval{\rho}{\hcall{\Tupk{x}}}{v}}
}
\PT{
	\AC{(\Eval{[\Map{x_i}{v_i}]}{e}{v_i'})^k_{i=1}}
	\RiLa{(\rho(y)=\Seqk{v})}
	\UC{\Eval{\rho}{\Comp{e}{x}{y}{\cdot}}{\Seqk{v'}}}
}\\[1ex]

\item Auxiliary \Jug{\Eval{}{\EvalF{\Tupk{v}}}{v}}

\PT{\Axiom{\Eval{}{\const{n}}{n}}}
\PT{\Axiom{\Eval{}{\iota{n}}{\{0,1,...,n-1\}}}}
\PT{\AC{} 
	\RiLa{(n_3= n_1+n_2)} 
	\UC{\Eval{}{\plus{n_1}{n_2}}{n_3}}}

\end{itemize}

\subsection{SVCODE syntax}
Stream id: $$\s \in \*{N} = \{0,1,2...\}$$
Stream tree: $$ \st ::= \s \ | \ (\st_1,\s) $$
SVCODE expressions: $$\varphi ::= \ctrl \ | \ \consta{a} \ | \ \toflag
\ | \ \usum \ | \ \maptwo \ | \ \scan $$

SVCODE program: 
\begin{align*}
	p ::= & \ \epsilon \\
	     &| \ \sdef{\s}{\psi(s_1,...,s_i)} \\
	     &| \ \withctrl{\s}{p}{\st} \\
	     &| \ p_1;p_2 	 
\end{align*}
%$$ p ::= \epsilon \ | \ \s:= \psi(s_1,...,s_i) \ | \  \withctrl{\s}{p}{\st} \ | \ p_1;p_2 $$

Target language values: 
$$b \in \{\T,\F \}$$
$$a ::= n \ | \ b \ | \ \unit $$
$$\v ::= \vrange{a_1}{a_i} $$

Define some operations on $\v$ for convenience:
\begin{itemize}
	\item $\vapp{\v_1}{\v_2} $: append $\v_2$ to $\v_1$
	\item $\tail{\langle a_1, a_2,...,a_i \rangle} = \vrange{a_2}{a_i}$
\end{itemize}


\subsection{SVCODE semantics}
%TODO add definition for empty streams mapping
\begin{itemize}
\item \Jug{\seval{p}{\sgm}{\v_c}{\sgm'}}

\PT{\Axiom{\seval{\epsilon}{\sgm}{\v_c}{\sgm}}}
\PT{\AC{\sevalf{\v_1}{\v_k}{\v_c}{\v}}
	\RiLa{((\sgm(\s_i) = \v_i)^k_{i=1})}
	\UC{\seval{\sdef{\s}{\phi\Tupk\s}}{\sgm}{\v_c}{\sgm[\Map{\s}{\v}]}}
} \\[2ex]

\PT{
	\AC{}
	\RiLa{(\sgm(\s)= \emptyv)}
	\UC{\seval{\withctrl{\s}{p}{\st}}{\sgm}{\v_c}{\sgm[\Maps{\st}{\emptyv}]}}
}\\[2ex]

\PT{
	\AC{\seval{p}{\sgm}{\v_s}{\sgm'}}
	\RiLa{(\sgm(\s)= \v_s = \vrange {a_1} {a_i})}
	\UC{\seval{\withctrl s p \st } \sgm {\v_c} \sgm'}
}\\[2ex]

\PT{
	\AC{\seval{p_1}{\sgm}{\v_c}{\sgm''}}
    \AC{\seval{p_2}{\sgm''}{\v_c}{\sgm'}}	
    \BC{\seval{p_1;p_2}{\sgm}{\v_c}{\sgm'}}
}


\item Auxiliary \Jug{\sevalf{\v_1}{\v_k}{\v_c}{\v}}

\PT{
	\AC{\block{\v_{11}}{\v_{k1}}{\v_1}}
	\AC{\sevalf{\v_{12}}{\v_{k2}}{\tail{\v_c}}{\v_2}}
	\RiLa{(\v = \vapp{\v_1}{\v_2})}
	\BC{\sevalf{\vapp{\v_{11}}{\v_{12}}}{\vapp{\v_{k1}}{\v_{k2}}}{\v_c}{\v}}
}\\[2ex]
	
\PT{\Axiom{\sevalf {\v_1} {\v_k} \emptyv \emptyv}}
	

\item Auxiliary \Jug{\block{\v_1}{\v_k}{\v}}

\PT{\Axiom{ \consta{a} \da \singl{a}}}
\PT{\Axiom{\toflag(\singl{n}) \da \langle \F_1,...,\F_n,\T \rangle}}
\PT{\AC{}
	\RiLa{(n_3= n_1+n_2)}
	\UC{\maptwo(\singl{n_1}, \singl{n_2}) \da \singl{n_3}}
} \\

\PT{
	\AC{\unary{\singl \F}{\v_{k1}}{\v_1}}
	\AC{\block{\v_{12}}{\v_{k2}}{\v_2}}
	\RiLa{(\v = \vapp{\v_1}{\v_2})}
	\BC{\block{\vapp {\singl\F} {\v_{12}}} {\vapp {\v_{k1}} {\v_{k2}}}{\v}}
}\\[2ex]

\PT{
	\AC{\unary{\oT} {\v_k}{\v}}
	\UC{\block{\oT}{\v_k} \v}
}\\[2ex]

\PT{
	\AC{\unaryv{\singl \F}{\v_{k1}}{\singl{n_1}} {n_0} {n_0'}}
	\AC{\blockv{\v_{12}}{\v_{k2}}{\v_2}{n_0'}}
	\BC{\blockv{\vapp {\singl\F} {\v_{12}}} {\vapp {\v_{k1}} {\v_{k2}}}{\vapp {\singl{n_1}} {\v_2}} {n_0}}
}\\[2ex]

\PT{
	\AC{\unaryv \oT {\v_k} {\singl{n_1}} {n_0} {}}
	\UC{\blockv{\oT}{\v_k} {\singl{n_1}} {n_0}}
}




\item Auxiliary \Jug{\unary{\singl b}{\v_k}{\v}}

\PT{ \Axiom{\usum(\oF) \da \singl\unit}}
\PT{\Axiom{\usum(\oT) \da \emptyv }} \\[1ex]

\PT{\Axiom{\scan_{n_0}(\oF,{\singl{n}}) \da^{n_0+n} {\singl{n_0}} 
	}}
\PT{\Axiom{\scan_{n_0}(\oT, \emptyv) \da {\singl{n_0}} }}

\end{itemize}


\subsection{Translation}
\begin{itemize}
	
\item 
\PT{
	\AC{\Trans{\del}{e}{\s_0+1}{\s_1}{\sfun p {\st}}}
	\UC{\Trans{\del}{e}{\s_0}{\s_1}{\Talet{\sdef{\s_0}{\ctrl};p}{\st}}}
}\\[2ex]

\item \Jug{\Trans{\del}{e}{\s_0}{\s_1}{\sfun{p}{\st}}}

\PT{\AC{}
	\RiLa{(\del(x)= \st)}
	\UC{\Trans{\del}{x}{\s_0}{\s_0}{\sfun{p}{\st}}}
}
\PT{\AC{\Trans{\del}{e_1}{\s_0}{\s_0'}{\sfun{p_1}{\st_1}}}
	\AC{\Trans{\del[\Map{x}{\st_1}]}{e_2}{\s_0'}{\s_1}{\sfun {p_2} {\st}}}
	\BC{\Trans{\del}{\Let{x}{e_1}{e_2}}{\s_0}{\s_1}{\sfun {p_1;p_2} {\st}}}
}

\PT{
	\AC{\tranf{\st_1}{\st_k}{\s_0}{\s_1}{\sfun{p}{\st}}}
	\RiLa{((\del(x_i)=\st_i)^k_{i=1})}
	\UC{\Trans{\del}{\phi \Tupk{x}}{\s_0}{\s_1}{\sfun{p}{\st}}}
}

\PT{
	\AC{\Trans{[\Map{x}{\st_1}]}{e}{\s_0+1}{\s_1}{\sfun{p}{\st}}}
	\RiLa{(\del(y)=(\st_1,\s_2))}
	\UC{\Trans{\del}{\Comp{e}{x}{y}{\cdot}}{\s_0}{\s_1}{\sfun{\sdef{\s_0}{\usum(\s_2)}; \withctrl{\s_0}{p}{\st}}{\st}}}
}


\item Auxiliary \Jug{\tranf{\st_1}{\st_k}{\s_0}{\s_1}{\sfun{p}{\st}}}

\PT{
	\Axiom{\const{a} \xR{\s_0+1}{\s_0}
		\sfun{\sdef{\s_0}{\consta{a}}}{\s_0}  }
} \\ [6ex]

\PT{
	\AC{}
	\RiLa{\left( \begin{aligned}
		  \s_{i+1} & = \s_i + 1 \\
		  p= & \sdef{\s_0}{\toflag(\s)} ; \\ 
		     & \sdef{\s_1}{\usum(s_0)} ; \\
		     & \withctrl{\s_1}{\sdef{\s_2}{\consta{1}}}{\s_2}; \\
	         & \sdef{\s_3}{\scan(\s_0,s_2)}
	    \end{aligned}\right)
    }
	\UC{\iota{\s} \xR{\s_4}{\s_0} {\sfun{p} {(\s_3,\s_0)}}}
}\\[4ex]

\PT{
	\Axiom{\plus{\s_1}{\s_2} \xR{\s_0+1}{\s_0}
	    \sfun{\sdef{\s_0}{\maptwo(\s_1,\s_2)}}{\s_0}}
}


\end{itemize}


\subsection{Value representation}

\emph{TODO: define $\sgm(\st)$}

\Jug{\ValRep{\sgm}{v}{\tau}{\st}}

\PT{
	\AC{}
	\RiLa{(\sgm(\s)=\singl{n})}
	\UC{\ValRep{\sgm}{n}{\int}{\s}}
}
\PT{
	\AC{(\ValRep{\sgm}{n_i}{\int}{\s_i})^k_{i=1}}
	\RiLa{\left(
		\begin{aligned}
	        \sgm(\s)= & {\sgm(\s_1)} \appop {\sgm(\s_2)} \appop ...
		               \appop {\sgm(\s_k)}   \\
		    \sgm(\s') = & \langle \F_1,..., \F_k, \T \rangle
		\end{aligned}
		\right)}
	\UC{\ValRep{\sgm}{\Seqk{n}}{\tseq{\int}}{(\s,\s')}}
}\\[4ex]



\PT{
	\AC{(\ValRep{\sgm}{v_i}{\tau}{(\st_i,\s_i)})^k_{i=1}}
	\RiLa{\left(
		  \begin{aligned}
		  	& \tau \ne \int \\
		  	& \sgm(\st) = {\sgm(\st_1)} \appop {\sgm(\st_2)} \appop ...
		  	\appop {\sgm(\st_k)}   \\
		  	& \sgm(\s) = {\sgm(\s_1)} \appop {\sgm(\s_2)} \appop ...
		  	\appop {\sgm(\s_k)}   \\
		  	& \sgm(\s') = \langle \F_1,..., \F_k, \T \rangle
		  \end{aligned}
		\right)
	}
	\UC{\ValRep{\sgm}{\Seqk{v}}{\tseq{\tau}}{((st,s),s')}}
}

\subsection{Correctness proof}

\end{document}
