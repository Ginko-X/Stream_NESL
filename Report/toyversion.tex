\documentclass[a4paper]{article}

\usepackage{proof}
\usepackage{amsmath, amssymb, amsthm}
\usepackage{bussproofs}
\usepackage{color}
\usepackage{geometry}
\usepackage{enumerate}
 
\geometry{left=2.5cm,right=2.5cm,top=3cm,bottom=3cm}

\newtheorem{thm}[equation]{Theorem}
\newtheorem{lem}[equation]{Lemma}
\newtheorem{prop}[equation]{Proposition}
\newtheorem{cor}[equation]{Corollary}
\newtheorem{conj}[equation]{Conjecture}

%syntax shorthands
\def\Mov#1#2{\mathbf{mov} \ #1  \ #2 }
\def\Add#1#2#3{\mathbf{add} \ #1 \ #2 \ #3 }
\def\Fst#1{\mathbf{fst} (#1)}
\def\Snd#1{\mathbf{snd} (#1)}
\def\Int{\mathbf{Int}}
\def\Let#1#2#3{\mathbf{let} \ #1 = #2 \ \mathbf{in} \ #3}
\def\Gplus#1#2{\mathbf{gplus} (#1,#2)}
\def\*#1{\mathbf{#1}\ }


% symbol shorthands
\def\Eva{\downarrow}
\def\Ra{\Rightarrow}
\def\Env{\ \vdash\ } 
\def\Sgm{\sigma}

% special font
\def\MC{\mathcal{C}} 
\def\ME{\mathcal{E}} 
\def\MP{\mathcal{P}} 
\def\MT{\mathcal{T}} 
\def\MV{\mathcal{V}}



% Judgment
\def\Jug#1{Judgment \boxed{#1} \\[1ex]}

\def\Map#1#2{#1 \mapsto #2}

% Evaluation
\def\Eval#1#2#3{#1 \Env #2 \Eva #3 } 
\def\FEval#1#2{#1 \Eva #2}  % function evaluation

% target semantics 
\def\TaSem#1#2#3{\langle #1,#2 \rangle \Eva #3}  
\def\Talet#1#2{\mathtt{let} \ #1 \ \mathtt{in} \ #2}

% compiling/translation
\def\Trans#1#2#3#4#5{#1 \Env #2 \Ra^{#4}_{#5} #3} 
\def\TransP#1#2#3#4#5{\mathbf{transPlus} (#1,#2) \Ra^{#4}_{#5} #3}

%typing
\def\Type#1#2#3{#1 \Env #2 : #3 } 

%value representation
\def\ValRep#1#2#3#4{#1 \Env #2 \triangleright_{#4} #3}

% rule with a name
\def\ERule#1#2{\textsc{E-#1} : \vcenter{#2}}
\def\PRule#1#2{\textsc{P-#1} : \vcenter{#2}}
\def\CRule#1#2{\textsc{C-#1} : \vcenter{#2}}

% only rule name
\def\EName#1{\textsc{E-#1}}
\def\PName#1{\textsc{P-#1}}
\def\CName#1{\textsc{C-#1}}

%proof tree shorthands
\def\AC#1{\AxiomC{$#1$}}
\def\UC#1{\UnaryInfC{$#1$}}
\def\BC#1{\BinaryInfC{$#1$}}
\def\TC#1{\TrinaryInfC{$#1$}}
\def\DP{\DisplayProof}
\def\LeLa{\LeftLabel}
\def\PT#1{#1 \DisplayProof} 
%UnaryInfC with a deriviation name
\def\UCN#1#2{\AC{#1} \noLine \UC{#2}}


%color
\def\Blue{\color{blue}}



\title{Toy language formalization}

\begin{document}
\maketitle

\setcounter{section}{-1}

\section{Level-0}
\subsection{Source language:}

$$e ::= n \ |\ e_1 + e_2 \ $$
$$ ( n \in \*{Z} )$$


\subsection{Source language semantics:}

\Jug{e \Eva n}

$$
\ERule{Cons}{
\infer{n \Eva n}{}
}\qquad
\ERule{Plus}{\infer[(n_1+n_2= n_3)]{e_1 + e_2 \Eva n_3}{e_1 \Eva n_1 & e_2 \Eva n_2}
}\qquad
$$

\subsection{Target language:}
$$ r \in \*N = \{0,1,2,...\} $$
$$ s ::= \Mov{r}{n} \ |\ \Add{r_1}{r_2}{r_3}$$
$$ p ::= s \  |\ p_1;p_2 $$



\subsection{Target language semantics: }
Environment $\Sgm = [\Map{r_1}{n_1}, ..., \Map{r_i}{n_i}]$.


\Jug{\TaSem{p}{\Sgm}{\Sgm'}}. 
$$
\PRule{Mov}{\infer{\TaSem {\Mov{r}{n}} \Sgm {\Sgm[\Map{r}{n}]}}{}
}\qquad
$$
$$
\PRule{Add}{\infer[(\Sgm(r_2)=n_2, \Sgm(r_3)=n_3, n_2+n_3=n_1)]
{\TaSem {\Add{r_1}{r_2}{r_3}} \Sgm {\Sgm[\Map{r_1}{n_1}]}}{}
}\qquad
$$
$$
\PRule{Seq}{\infer{\TaSem {p_1;p_2} \Sgm {\Sgm_2}}
{\TaSem {p_1} \Sgm {\Sgm_1} & \TaSem {p_2} {\Sgm_1} {\Sgm_2}}
}\qquad
$$ 

\subsection{Translation:}
\Jug{e \Ra_{r_1}^{r_0} \Talet{p}{r}}
Newly generated register identifiers  start from $r_0$, end at (but not include) $r_1$.
$$
\CRule{Cons}{\infer{n \Ra^{r_0}_{r_0+1} \Talet{\Mov{r_0}{n}}{r_0}}{}
}\qquad
$$

$$
\CRule{Plus}{\infer{e_1+e_2 \Ra^{r_0}_{r_2'+1} \Talet{p_1;(p_2;\Add{r_2'}{r_1}{r_2})}{r_2'+1}}
{e_1 \Ra^{r_0}_{r_1'} \Talet{p_1}{r_1}  & e_2 \Ra^{r_1'}_{r_2'} \Talet{p_2}{r_2}}
}\qquad 
$$ \\[1ex]


\subsection{Correctness theorem:}

\begin{lem}
	If $e \Ra^{r_0}_{r_1} \Talet{p}{r}$, then $r_0 \le r_1$ and $r < r_1$.
	\label{fresh}
\end{lem}

\begin{thm}
 If $e \Eva n$ (by some derivation $\ME$), $e \Ra^{r_0}_{r_1} \Talet{p}{r}$ (by $\MC$), then $\TaSem{p}{\Sgm}{\Sgm'}$ (by $\MP$), $\forall r' < r_0.\Sgm'(r') = \Sgm(r')$, and $\Sgm'(r) = n$ .  
\end{thm}
 

\begin{proof}
 By induction on the syntax of $e$:
 \begin{itemize}
 	\item Case $e = n_0$, then $n = n_0$, by $\EName{Cons}$: $\ME$ = 
 	$\infer{n_0 \Eva n_0} {}$,
 	by $\CName{Cons}: \MC = $
 	$\infer{n_0 \Ra^{r_0}_{r_0+1} \Talet{\Mov{r_0}{n_0}}{r_0}}{}$,
 	so $p$ = $\Mov{r_0}{n_0}$, $r= r_0$. \\
  Then by $\PName{Mov}$, we get $\MP$ =
  $\infer{\TaSem {\Mov{r_0}{n_0}} \Sgm {\Sgm[\Map{r_0}{n_0}]}}{}$.\\ 
  Therefore we have
  $\forall r' < r_0. \Sgm[\Map{r_0}{n_0}](r') = \Sgm(r') $, and
   $\Sgm[\Map{r_0}{n_0}](r_0) = n_0 $ as required.
 	
  \item Case $e = e_1 + e_2$.\\
  By \EName{Plus}, $\ME$ must have the shape:   
    \PT{
    	\UCN{\ME_1}{e_1 \Eva n_1}
    	\UCN{\ME_2}{e_2 \Eva n_2}
    	\BC{ e_1 + e_2 \Eva n_1 + n_2}
    }, thus $n = n_1 + n_2 $. \\
  
  By \CName{Plus}, $\MC$ must have the shape:
  \PT{
  	\UCN{\MC_1}{e_1 \Ra^{r_0}_{r_1'} \Talet{p_1}{r_1}}
  	\UCN{\MC_2}{e_2 \Ra^{r_1'}_{r_2'} \Talet{p_2}{r_2}}
  	\BC{e_1+e_2 \Ra^{r_0}_{r_2'+1} \Talet{p_1;p_2;\Add{r_2'}{r_1}{r_2}}{r_2'}}
  } 

  So $p = p_1;p_2;\Add{r_2'}{r_1}{r_2}$, and $r = r_2'$.\\
  
  By IH on $\ME_1$ with $\MC_1$, we get $\MP_1 =  \TaSem{p_1}{\Sgm}{\Sgm_1}$ for some  $\Sgm_1$, $\forall r' < r_0. \Sgm_1(r') = \Sgm(r')$, and $\Sgm_1(r_1) = n_1$. \\

  Likewise, by IH on  $\ME_2$ with $\MC_2$, we get $ \MP_2 = \TaSem{p_2}{\Sgm_1}{\Sgm_2}$ for some $\Sgm_2$, $\forall r'' < r_1'. \Sgm_2(r'') = \Sgm_1(r'')$,  and $\Sgm_2(r_2) = n_2$.\\ 
  
  By Lemma \ref{fresh} on $\MC_1$, $r_0 \le r_1'$, and $r_1 < r_1'$. Since $r_0 \le r_1'$, we get $\forall r''' < r_0. \Sgm_2(r''') = \Sgm_1(r''') = \Sgm(r''')$; since $r_1 < r_1'$, we get $\Sgm_2(r_1) = \Sgm_1(r_1) = n_1$. \\
  
  Use $\PName{Seq}$ and $\PName{Add}$, we construct: \\
  \PT{
  	\UCN{\MP_1}{\TaSem{p_1}{\Sgm}{\Sgm_1}}
  		\UCN{\MP_2}{\TaSem{p_2}{\Sgm_1}{\Sgm_2}} 
  		\RightLabel{$(\Sgm_2(r_2)=n_2, \Sgm_2(r_1) = n_1)$}
  		\AC{}
  		\UC{\TaSem{\Add{r_2'}{r_1}{r_2}}{\Sgm_2}{\Sgm_2[\Map{r_2'}{n_1+ n_2}]}}
  	\BC{\TaSem{p_2;\Add{r_2'}{r_1}{r_2}}{\Sgm_1}{\Sgm_2[\Map{r_2'}{n_1+ n_2}]}}
    \BC{\TaSem{p_1;(p_2;\Add{r_2'}{r_1}{r_2})}{\Sgm}{{\Sgm_2[\Map{r_2'}{n_1+ n_2}]}}}
  }\\
  
  Therefore, $\Sgm_2[\Map{r_2'}{n_1+ n_2}](r_2') =  n_1 + n_2 = n$. Take $\Sgm' = \Sgm_2$ and we are done.
   

 \end{itemize}
\end{proof}

\hspace{1cm}


\section{Level-1}
\subsection{Extended source language}
$$e ::= ... \ |\ x \ |\ \Let{x}{e_1}{e_2} $$

\subsection{Extended semantics: }
High-level runtime environment $\rho = [\Map{x_1}{n_1}, ..., \Map{x_i}{n_i} ]$. \\

\Jug{\Eval{\rho}{e}{n}}
$$
\infer[(\rho(x)=n)]{\Eval{\rho}{x}{n}}{}
\qquad
\infer{\Eval{\rho} {\Let{x}{e_1}{e_2}} {n} }
{\Eval{\rho}{e_1}{ n_1} & \Eval{\rho[\Map{x}{n_1}]} {e_2} {n} }
\qquad
$$

\subsection{Target language:}
(added $\epsilon$ to $p$)
$$ r \in \*N = \{0,1,2,...\} $$
$$ s ::= \Mov{r}{n} \ |\ \Add{r_1}{r_2}{r_3}$$
$$ p ::= \epsilon \ | \  s \  |\ p_1;p_2 $$

\subsection{Extended target language semantics: }
\Jug{\TaSem{p}{\Sgm}{\Sgm'}}
$$
\infer{\TaSem {\epsilon} \Sgm {\Sgm}}{}
\qquad
$$


\subsection{Extended translation:}
Translation environment $\delta = [\Map{x_1}{r_1}, ..., \Map{x_i}{r_i} ]$.\\

\Jug{ \Trans{\delta}{e}{\Talet{p}{r}}{r_0}{r_1}}
$$
\infer[(\delta(x)=r)]{\Trans{\delta}{x}{\Talet{\epsilon}{r}}{r_0} {r_0}}{}
$$
$$
\infer{\Trans{\delta}{\Let{x}{e_1}{e_2}}{\Talet{p_1;p_2}{r_2}} {r_0} {r_2'} }
	{\Trans{\delta}{e_1}{\Talet{p_1}{r_1}} {r_0} {r_1'} & \Trans{\delta[\Map{x}{r_1}]}{e_2}{\Talet{p_2}{r_2}} {r_1'} {r_2'}} 
$$


\subsection{Correctness theorem: }

\begin{lem}
	If $\Trans{\delta}{e}{\Talet{p}{r}}{r_0}{r_1}$, then $r_0 \le r_1$ and $r < r_1$.
	\label{fresh1}
\end{lem}

\begin{thm}
	If $\Eval{\rho}{e}{n}$, $\Trans{\delta}{e}{\Talet{p}{r}} {r_0} {r'}$, and $\forall x \in dom(\rho).\rho(x) = \Sgm(\delta(x))$,
	then $\TaSem{p}{\Sgm}{\Sgm'}$, $\forall r' < r_0.\Sgm'(r') = \Sgm(r') $, and $\Sgm'(r) = n$. 
\end{thm}


\section{Level-2}
\subsection{Extended source language:}
$$e ::= ... \ |\ (e_1,e_2) \ |\ \Fst{e} \ |\ \Snd{e}$$

\subsection{Added values:}

$$v ::= n \ | \ (v_1,v_2) $$

\subsection{Added source language type system:}
$$ \tau ::= \Int \ | \ (\tau_1,\tau_2) $$

Type environment $\Gamma = [\Map{x_1}{\tau_1}, ..., \Map{x_i}{\tau_i} ]$.

\begin{itemize}

\item \Jug{\Type{\Gamma}{e}{\tau}}
$$
\infer{\Type{\Gamma}{n}{\Int}}{}
\qquad
\infer{\Type{\Gamma}{e_1+e_2}{\tau}}
{\Type{\Gamma}{e_1}{\tau} & \Type{\Gamma}{e_2}{\tau}}
$$

$$
\infer[(\Gamma(x)=\tau)]{\Type{\Gamma}{x}{\tau}}{}
\qquad
\infer{\Type{\Gamma}{\Let{x}{e_1}{e_2}}{\tau}}
{\Type{\Gamma}{e_1}{\tau_1} & \Type{\Gamma[\Map{x}{\tau_1}]}{e_2}{\tau}}
$$

$$
\infer{\Type{\Gamma}{(e_1,e_2)}{(\tau_1, \tau_2)}}
{\Type{\Gamma}{e_1}{\tau_1} & \Type{\Gamma}{e_2}{\tau_2}} 
\qquad
\infer{\Type{\Gamma}{\Fst{e}}{\tau_1}}
{\Type{\Gamma}{e}{(\tau_1,\tau_2)}}
\qquad
\infer{\Type{\Gamma}{\Snd{e}}{\tau_2}}
{\Type{\Gamma}{e}{(\tau_1,\tau_2)}}
$$

\item \Jug{\Type{}{v}{\tau}}
$$
\infer{\Type{}{n}{\Int}}{}
\qquad
\infer{\Type{}{(v_1,v_2)}{(\tau_1,\tau_2)}}
{\Type{}{v_1}{\tau_1} & \Type{}{v_2}{\tau_2}}
$$ 

\item  Auxiliary \Jug{\Type{}{\*{gplus}(v_1,v_2)}{\tau}}
\indent (general plus operation typing rules)
$$\infer{\Type{}{\*{gplus}(n_1,n_2)}{\Int}}{}
\qquad
\infer{\Type{}{\*{gplus}((v_{10},v_{11}),(v_{20},v_{21}))}{(\tau_1,\tau_2)}}
{\Type{}{\*{gplus}(v_{10},v_{20})}{\tau_1} & {\Type{}{\*{gplus}(v_{11},v_{21})}{\tau_2}}} 
$$

\end{itemize}


\subsection{Extended semantics: }

\Jug{\rho \Env e \Eva v }
(fixed runtime environment $\rho = [\Map{x_1}{v_1}, ..., \Map{x_i}{v_i} ]$) 
$$
\infer{\Eval{\rho}{n}{n}}{}
\qquad
\infer{\Eval{\rho}{e_1+e_2}{v_3}}
{\Eval{\rho}{e_1}{v_1} & \Eval{\rho}{e_2}{v_2} & \FEval{\Gplus{v_1}{v_2}}{v_3}}
$$

$$
\infer[(\rho(x)=v)]{\rho \Env x \Eva v}{}
\qquad
\infer{\rho \Env \Let{x}{e_1}{e_2} \Eva v }
{\rho \Env e_1 \Eva v_1 & \rho[\Map{x}{v_1}] \Env e_2 \Eva v }
$$

$$
\infer{\Eval{\rho}{(e_1,e_2)}{(v_1,v_2)}}
{\Eval{\rho}{e_1}{v_1} & {\Eval{\rho}{e_2}{v_2}}}
\qquad
\infer{\Eval{\rho}{\Fst{e}}{v_1}}
{\Eval{\rho}{e}{(v_1,v_2)}}
\qquad
\infer{\Eval{\rho}{\Snd{e}}{v_2}}
{\Eval{\rho}{e}{(v_1,v_2)}}
$$ \\[1ex]

Auxiliary \Jug{\FEval{\Gplus{v_1}{v_2}}{v_3}}
$$
\infer[(n_1+n_2=n_3)]{\FEval{\Gplus{n_1}{n_2}}{n_3}}{}
\qquad
\infer{\FEval{\Gplus{(v_{10},v_{11})}{(v_{20},v_{21})}}{(v_{30},v_{31})}}
{\FEval{\Gplus{v_{10}}{v_{20}}}{v_{30}}  & \FEval{\Gplus{v_{11}}{v_{21}}}{v_{31}}}
\qquad
$$ \\[1ex]

\subsection{Target language:}
$$ rs ::= r \ | \ (rs_1,rs_2) $$
$s$, $p$ and semantics no change. \\

Define a function $rset$ to convert $rs$ to a set of $r$: \\
\indent $rset(r) = \{r\}$ \\
\indent $rset((rs_1,rs_2)) = rset(rs_1) \cup rset(rs_2)$


\subsection{Extended translation:}

\Jug{\Trans{\delta}{e}{\Talet{p}{rs}}{r_0} {r_1} }
(fixed $\delta = [\Map{x_1}{rs_1}, ..., \Map{x_i}{rs_i} ]$) \\[1ex]
$$
\infer{\Trans{\delta}{e_1+e_2}{\Talet{p1;(p2;p3)}{rs3}} {r} {r_3}}
{\Trans{\delta}{e_1}{\Talet{p_1}{rs_1}} {r} {r_1} & \Trans{\delta}{e_2}{\Talet{p_2}{rs_2}} {r_1} {r_2} & \TransP{rs_1}{rs_2}{\Talet{p_3}{rs_3}} {r_2} {r_3} }
$$
$$
\infer{\Trans{\delta}{\Let{x}{e_1}{e_2}}{\Talet{p_1;p_2}{rs_2}} {r_0}{r_2}} 
{\Trans{\delta}{e_1}{\Talet{p_1}{rs_1}}{r_0}{r_1} & \Trans{\delta[\Map{x}{rs_1}]}{e_2}{\Talet{p_2}{rs_2}} {r_1} {r_2} } 
$$

$$
\infer{\Trans{\delta}{(e_1,e_2)}{\Talet{p_1;p_2}{(rs_1,rs_2)}} {r_0} {r_2} }
{\Trans{\delta}{e_1}{\Talet{p_1}{rs_1}} {r_0} {r_1} & \Trans{\delta}{e_2}{\Talet{p_2}{rs_2}} {r_1} {r_2}}
$$

$$
\infer{\Trans{\delta}{\Fst{e}}{\Talet{p_1}{rs_1}} {r_0} {r_1}}
{\Trans{\delta}{e}{\Talet{p_1}{(rs_1,rs_2)}} {r_0}{r_1}}
\qquad
\infer{\Trans{\delta}{\Snd{e}}{\Talet{p_1}{rs_2}} {r_0} {r_1}}
{\Trans{\delta}{e}{\Talet{p_1}{(rs_1,rs_2)}} {r_0}{r_1}}
$$ \\[1ex]


Auxiliary \Jug{ \TransP{rs_1}{rs_2}{\Talet{p}{rs_3}} {r_0} {r_1}}
$$
\infer{\TransP{r_1}{r_2}{\Talet{\Add{r_3}{r_1}{r_2}}{r_3}} {r_3}{r_3+1}}{}
$$
$$
\infer{\TransP{(rs_{10},rs_{11})}{(rs_{20},rs_{21})}{\Talet{p_1;p_2}{(rs_{30},rs_{31})}} {r_0} {r_2}}
{\TransP{rs_{10}}{rs_{20}}{\Talet{p_1}{rs_{30}}} {r_0} {r_1} & \TransP{rs_{11}}{rs_{21}}{\Talet{p_2}{rs_{31}}} {r_1} {r_2} } 
$$

\subsection{Value representation:}
\Jug{\ValRep{\Sgm}{v}{rs}{\tau}}
($v: \tau$ can be represented as $rs$ in $\Sgm$)
$$
\infer[(\Sgm(r)=n)]{\ValRep{\Sgm}{n}{r}{\Int}}{}
\qquad
\infer{\ValRep{\Sgm}{(v_1,v_2)}{(rs_1,rs_2)}{(\tau_1,\tau_2)}}
{\ValRep{\Sgm}{v_1}{rs_1}{\tau_1} & {\ValRep{\Sgm}{v_2}{rs_2}{\tau_2}}}
\qquad
$$




\subsection{Correctness theorem:}
\begin{lem}
	If $\TransP{rs_1}{rs_2}{\Talet{p}{rs_3}}{r_0}{r_1}$, then $r_0 \le r_1$ and $\forall r \in rset(rs_3). r < r_1$.
	\label{trans-fresh}
\end{lem}

\begin{lem}
	If $\Trans{\delta}{e}{\Talet{p}{rs}}{r_0}{r_1}$, then $r_0 \le r_1$ and $\forall r \in rset(rs).r < r_1$.
	\label{fresh2}
\end{lem}


\begin{lem}
   If \begin{enumerate}[(i)]
   	\item $\ValRep{\Sgm}{v_1}{rs_1}{\tau}$, and $\ValRep{\Sgm}{v_2}{rs_2}{\tau}$
   	\item $\FEval{\Gplus{v_1}{v_2}}{v_3}$
   	\item $\TransP{rs_1}{rs_2}{\Talet{p}{rs_3}} {r_0} {r_1}$
   \end{enumerate}
   then 
   \begin{enumerate}[(i)]
   	\item $\TaSem{p}{\Sgm}{\Sgm'}$ 
   	\item $\ValRep{\Sgm'}{v_3}{rs_3}{\tau}$
   	\item $\forall r < r_0. \Sgm'(r) = \Sgm(r)$ 
   \end{enumerate}
\label{plus-thm}   
\end{lem}

\begin{thm}
	If 
	\begin{enumerate}[(i)]
		\item $\Type{\Gamma}{e}{\tau}$ (by some derivation $\MT$),
		\item $\Eval{\rho}{e}{v}$ (by $\ME$), and $\forall x \in dom(\Gamma). \rho(x):\Gamma(x)$, 
		\item $\Trans{\delta}{e}{\Talet{p}{rs}}{r_0}{r_1}$ (by $\MC$),
		and $\forall x \in dom(\Gamma).\ValRep{\Sgm}{\rho(x)}{\delta(x)}{\Gamma(x)}$
	\end{enumerate}
	then 
	\begin{enumerate}[(i)]
		\item $\TaSem{p}{\Sgm}{\Sgm'}$ (by $\MP$),
		\item  $\ValRep{\Sgm'}{v}{rs}{\tau}$ (by $\MV$)
		\item $\forall r < r_0.\Sgm'(r) = \Sgm(r) $. 
	\end{enumerate} 
\end{thm}


 
\begin{proof}
By induction on the syntax of $e$.
	\begin{itemize}
	\item Case $e = e_1 + e_2$.\\
	Then must have: 
	\PT{
		\UCN{\MT_1}{\Type{\Gamma}{e_1}{\tau}}
		\UCN{\MT_2}{\Type{\Gamma}{e_2}{\tau}}
		\LeLa{$\MT =$} 
		\BC{\Type{\Gamma}{e_1+e_2}{\tau}}
    }
	
	\PT{
		\UCN{\ME_1}{\Eval{\rho}{e_1}{v_1}}
		\UCN{\ME_2}{\Eval{\rho}{e_2}{v_2}}
		\UCN{\ME_3}{\FEval{\Gplus{v_1}{v_2}}{v}}
		\LeLa{$\ME =$} 
		\TC{\Eval{\rho}{e_1+e_2}{v}}
	}
	
\makebox[0.9\textwidth][c] {
	\PT{		
		\UCN{\MC_1}{\Trans{\delta}{e_1}{\Talet{p_1}{rs_1}}{r_0}{r_1}}
		\UCN{\MC_2}{\Trans{\delta}{e_2}{\Talet{p_2}{rs_2}}{r_1}{r_2}}
		\UCN{\MC_3}{\TransP{rs_1}{rs_2}{\Talet{p_3}{rs_3}}{r_2}{r_3}}
		\LeLa{$\MC=$}
		\TC{\Trans{\delta}{e_1+e_2}{\Talet{p1;(p2;p3)}{rs_3}}{r_0}{r_3}}
	}
}

So $p= p1;p2;p3, rs=rs_3$.\\
By IH on $\MT_1,\ME_1,\MC_1$, we get $\MP_1$ of $\TaSem{p_1}{\Sgm}{\Sgm_1}$, $\MV_1$ of $ \ValRep{\Sgm_1}{v_1}{rs_1}{\tau}$, and $\forall r < r_0. \Sgm_1(r) = \Sgm(r)$.\\
Likewise, by IH on $\MT_2,\ME_2,\MC_2$, we get $\MP_2$ of $\TaSem{p_2}{\Sgm_1}{\Sgm_2}$, $\MV_2$ of $ \ValRep{\Sgm_2}{v_2}{rs_2}{\tau}$, and $\forall r' < r_1. \Sgm_2(r') = \Sgm_1(r')$.\\

By theorem \ref{plus-thm} on $\MV_1,\MV_2, \ME_3,\MC_3$ , we get $\MP_3$ of $\TaSem{p_3}{\Sgm_2}{\Sgm_3}$, $\ValRep{\Sgm_3}{v}{rs_3}{\tau}$, and $\forall r'' < r_2. \Sgm_3(r'') = \Sgm_2(r’‘)$. \\

Then we can construct:\\

	\PT{
	\UCN{\MP_1}{\TaSem{p_1}{\Sgm}{\Sgm_1}}
	\UCN{\MP_2}{\TaSem{p_2}{\Sgm_1}{\Sgm_2}}
	\UCN{\MP_3}{\TaSem{p_3}{\Sgm_2}{\Sgm_3}}
    \BC{\TaSem{p_2;p_3}{\Sgm_1}{\Sgm_3}}
    \BC{\TaSem{p_1;(p_2;p_3)}{\Sgm}{\Sgm_3}}
	}	
    \\[1ex]
    
By lemma \ref{fresh2} on $\MC_1$, we get $r_0 \le r_1$. Similarly, we also get $r_1 \le r_2$ by lemma \ref{fresh2} on $MC_2$. 
Therefore, $r_0 \le r_1 \le r_2$, hence $\forall r''' < r_0. \Sgm_3(r''') = \Sgm_2(r''') = \Sgm_1(r''') = \Sgm(r''')$. 
 
 Take $\Sgm' = \Sgm_3$ and we are done.
 
 \item Case $e=(e_1,e_2).$\\
 Must have:
 	\PT{
 		\UCN{\MT_1}{\Type{\Gamma}{e_1}{\tau_1}}
 		\UCN{\MT_2}{\Type{\Gamma}{e_2}{\tau_2}}
		\LeLa{$\MT =$} 
 		\BC{\Type{\Gamma}{(e_1,e_2)}{(\tau_1,\tau_2)}}
    }
 
 	\PT{
 		\UCN{\ME_1}{\Eval{\rho}{e_1}{v_1}}
 		\UCN{\ME_2}{\Eval{\rho}{e_2}{v_2}}
 		\LeLa{$\ME =$} 
 		\BC{\Eval{\rho}{(e_1,e_2)}{(v_1,v_2)}}
 	} 
 	
 	\PT{
 		\UCN{\MC_1}{\Trans{\delta}{e_1}{\Talet{p_1}{rs_1}}{r_0}{r_1}}
 		\UCN{\MC_2}{\Trans{\delta}{e_2}{\Talet{p_2}{rs_2}}{r_1}{r_2}}
 		\LeLa{$\MC=$}
 		\BC{\Trans{\delta}{(e_1,e_2)}{\Talet{p1;p2}{(rs_1,rs_2)}}{r_0}{r_2}}
 	} \\[1ex]
 
 So $\tau = (\tau_1,\tau_2), v = (v_1,v_2), rs = (rs_1,rs_2)$.\\
 By IH on $\MT_1,\ME_1,\MC_1$, we get $\MP_1$ of $\TaSem{p_1}{\Sgm}{\Sgm_1}$, $\MV_1$ of$ \ValRep{\Sgm_1}{v_1}{rs_1}{\tau_1}$, and $\forall r < r_0. \Sgm_1(r) = \Sgm(r)$.\\
 Likewise, by IH on $\MT_2,\ME_2,\MC_2$, we get $\MP_2$ of $\TaSem{p_2}{\Sgm_1}{\Sgm_2}$, $\MV_2$ of $ \ValRep{\Sgm_2}{v_2}{rs_2}{\tau_2}$, and  $\forall r' < r_1. \Sgm_2(r') = \Sgm_1(r')$.\\
 
 Then we can construct:\\
 
 \PT{
 	\UCN{\MP_1}{\TaSem{p_1}{\Sgm}{\Sgm_1}}
 	\UCN{\MP_2}{\TaSem{p_2}{\Sgm_1}{\Sgm_2}} 
 	\BC{\TaSem{p_1;p_2}{\Sgm}{\Sgm_2}}
 }\\[1ex]

 By lemma \ref{fresh2} on $\MC_1$, we have $r_0 \le r_1$, and $\forall r \in rset(rs_1). r < r_1$. \\
 Since $r_0 \le r_1$, we get $\forall r'' < r_0. \Sgm_2(r'') = \Sgm_1(r'') = \Sgm(r'')$. \\ 
 Since $\forall r \in rset(rs_1). r < r_1$, then $\forall r \in rset(rs_1). \Sgm_2(r) = \Sgm_1(r)$. Therefore, there exists some $\MV_1'$ of  $\ValRep{\Sgm_2}{v_1}{rs_1}{\tau_1}$.\\
 
 Then we can construct:
 \PT{
 	\UCN{\MV_1'}{\ValRep{\Sgm_2}{v_1}{rs_1}{\tau_1}}
 	\UCN{\MV_2}{\ValRep{\Sgm_2}{v_2}{rs_2}{\tau_2}}
 	\BC{\ValRep{\Sgm_2}{(v_1,v_2)}{(rs_1,rs_2)}{(\tau_1,\tau_2)}}
 } \\[1ex]
 
 Take $\Sgm' = \Sgm_2$ and we are done.
 

\item Case $e = \Let{x}{e_1}{e_2}$. \\[1ex]
 Must have:
 \PT{
	\UCN{\MT_1}{\Type{\Gamma}{e_1}{\tau_1}}
	\UCN{\MT_2}{\Type{\Gamma[\Map{x}{\tau_1}]}{e_2}{\tau}}
	\LeLa{$\MT =$} 
	\BC{\Type{\Gamma}{\Let{x}{e_1}{e_2}}{\tau}}
 	}

 \PT{	
	\UCN{\ME_1}{\Eval{\rho}{e_1}{v_1}}
	\UCN{\ME_2}{\Eval{\rho[\Map{x}{v_1}]}{e_2}{v}}
	\LeLa{$\ME =$} 
	\BC{\Eval{\rho}{\Let{x}{e_1}{e_2}}{v}}
	} 
	
 \PT{
	\UCN{\MC_1}{\Trans{\delta}{e_1}{\Talet{p_1}{rs_1}}{r_0}{r_1}}
	\UCN{\MC_2}{\Trans{\delta[\Map{x}{rs_1}]}{e_2}{\Talet{p_2}{rs}}{r_1}{r_2}}
	\LeLa{$\MC=$}
	\BC{\Trans{\delta}{\Let{x}{e_1}{e_2}}{\Talet{p_1;p_2}{rs}}{r_0}{r_2}}
    } \\[1ex]

 So $p = p_1;p_2$. \\
 By IH on $\MT_1,\ME_1,\MC_1$, we get $\MP_1$ = $\TaSem{p_1}{\Sgm}{\Sgm_1}$, $\MV_1 = \ValRep{\Sgm_1}{v_1}{rs_1}{\tau_1}$, and $\forall r < r_0. \Sgm_1(r) = \Sgm(r)$.\\
 
 Since from $\MV_1$ we know $v_1:\tau_1$, then $\rho[\Map{x}{v_1}](x) : \Gamma[\Map{x}{\tau_1}](x)$ holds, and \\ $\ValRep{\Sgm_1}{\rho[\Map{x}{v_1}](x)}{\delta[\Map{x}{rs_1}](x)}{\Gamma[\Map{x}{\tau_1}](x)}$ also holds. \\
 
 Then by IH on  $\MT_2,\ME_2,\MC_2$, we get $\MP_2$ = $\TaSem{p_2}{\Sgm_1}{\Sgm_2}$, $\MV_2 = \ValRep{\Sgm_2}{v}{rs}{\tau}$, and $\forall r' < r_1. \Sgm_2(r') = \Sgm_1(r')$.\\
 
 So we can construct
  \PT{
 	 \UCN{\MP_1}{\TaSem{p_1}{\Sgm}{\Sgm_1}}
 	 \UCN{\MP_2}{\TaSem{p_2}{\Sgm_1}{\Sgm_2}} 
  	 \BC{\TaSem{p_1;p_2}{\Sgm}{\Sgm_2}}
  }\\

 By lemma \ref{fresh2} on $\MC_1$: $r_0 \le r_1$. Therefore, $\forall r'' < r_0. \Sgm_2(r'') = \Sgm_1(r'') = \Sgm(r'')$.
 
 Take $\Sgm' = \Sgm_2$ and we are done. 

\item  Case $e = n$.

Must have $\MT = \infer{\Type{\Gamma}{n}{\Int}}{}$, $\ME =\infer{ \Eval{\rho}{n}{n}}{}$, and $\MC = \infer{\Trans{\delta}{n}{\Talet{\Mov{r_0}{n}}{r_0}}{r_0}{r_0+1}}{}$. \\
So $p = \Mov{r_0}{n}, rs= r_0, v=n$, and $\tau = \Int$. \\
Then immediately we get $\infer{\TaSem{\Mov{r_0}{n}}{\Sgm}{\Sgm[\Map{r_0}{n}]}}{}$, 
$\ValRep{\Sgm[\Map{r_0}{n}]}{n}{r_0}{\Int}$, \\
and $\forall r < r_0. \Sgm[\Map{r_0}{n}](r) = \Sgm(r)$ as required. 

\item Case $e_=x$. 

Must have $\MT = \infer[(\Gamma(x)= \tau)]{\Type{\Gamma}{x}{\tau}}{}$,
$\ME = \infer[(\rho(x)=v)]{\Eval{\rho}{x}{v}}{}$, \\
and $\MC = \infer[(\delta(x)=rs)]{\Trans{\delta}{x}{\Talet{\epsilon}{rs}}{r_0}{r_0}}{}$. \\
So $p= \epsilon.$ \\
Immediately we get $\TaSem{\epsilon}{\Sgm}{\Sgm}$, $\ValRep{\Sgm}{v}{rs}{\tau}$, and $\forall r < r_0. \Sgm(r) = \Sgm(r)$ as required.

\item Case $e = \Fst{e_1}$.

Must have: 
\PT{
	\UCN{\MT_1}{\Type{\Gamma}{e_1}{(\tau_1,\tau_2)}}
	\LeLa{$\MT = $}
	\UC{\Type{\Gamma}{\Fst{e_1}}{\tau_1}}
} for some $\tau_2$,

\PT{
	\UCN{\ME_1}{\Eval{\rho}{e_1}{(v_1,v_2)}}
	\LeLa{$\ME = $}
	\UC{\Eval{\rho}{\Fst{e_1}}{v_1}} 
} for some $v_2$, 


\PT{
	\UCN{\MC_1}{\Trans{\delta}{e_1}{\Talet{p}{(rs_1,rs_2)}}{r_0}{r_1}}
	\LeLa{$\MC = $}
	\UC{\Trans{\delta}{\Fst{e_1}}{\Talet{p}{rs_1}}{r_0}{r_1}}
} for some $rs_2$. \\

So $\tau = \tau_1, v = v_1, rs = rs_1. $\\
By IH on $\MT_1, \ME_1, \MC_1$, we get $\MP$ of $\TaSem{p}{\Sgm}{\Sgm_1}$, 
$\MV_1$ of $\ValRep{\Sgm}{(v_1,v_2)}{(rs_1,rs_2)}{(\tau_1,\tau_2)}$,\\
and $\forall r < r_0. \Sgm_1(r) = \Sgm(r)$. \\
$\MV_1$ must have the shape: 
	\PT{
		\UCN{\MV}{\ValRep{\Sgm_1}{v_1}{rs_1}{\tau_1}}
		\UCN{}{\ValRep{\Sgm_1}{v_2}{rs_2}{\tau_2}}
		\BC{\ValRep{\Sgm_1}{(v_1,v_2)}{(rs_1,rs_2)}{(\tau_1,\tau_2)}}
	}

So now we have $\MV$. Take $\Sgm' = \Sgm_1$ and we are done. 

\item  The case  where $ e = \Snd{e_1}$ is analogous to the case above.

\end{itemize}

\end{proof}

\end{document}