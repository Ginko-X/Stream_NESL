\chapter{Introduction}

\section{Background}
\subsection{Nested data parallelism}

\section{NESL}
%NESL \cite{blel95nesl} 

\subsection{Work-depth cost model}

% \cite{blel96cost}

\section{SNESL}
Streaming NESL (SNESL) \cite{Fphd} is a refinement of NESL
that attempts to imporve the efficiency of space usage. 
It extends NESL with two features: streaming semantics and a cost model for space usage. 
The basic idea behind the streaming semantics may be described as:
data-parallelism can be realized not only in terms of space, as NESL has demonstrated, but also, for some restricted cases, in terms of time. 
When there is no enough space to store all the data at the same time, computing them chunk by chunk may be a way out.

\subsection{Type system}

The types of a minimalistic version of SNESL defined in \cite{Fphd} are:
\begin{align*} 
& \pi ::= \bool \ | \ \int \ | \ ...  \\
& \varphi ::= \pi \ | \ (\varphi_1,...,\varphi_k) \ | \ [\varphi]  \\
& \tau ::= \varphi \ | \ (\tau_1,...,\tau_k) \ | \ \tseq{\tau}  
\end{align*}
Here $\pi$ stands for the primitive types and $\varphi$ the concrete types, both originally supported in NESL.
The type $[\varphi]$, which is called $sequences$ in NESL and $vectors$ in SNESL, represents spatial collections of homogeneous data, and must be
fully allocated or $materialized$ in memory at once.  $(\varphi_1,...,\varphi_k)$ are tuples with $k$ components that may be of different types.

The novel extension is the $streamable$ types $\tau$, which generalizes the types of data that are not necessary to be completely materialized at once, but rather in a streaming fashion. 
In particular, the type $\{\tau\}$, called $squences$ in SNESL, represents collections of data computed in terms of time.
So, even with a small size of memory, SNESL could execute programs which is impossible in NESL due to space limitation or more space efficiently than in NESL. 

For clearity, from now on, we will use the terms consistent with SNESL.

\subsection{Values and expressions}

The values of SNESL are as follows:
\begin{align*}
& a ::=  \T \ | \ \F \ | \ n \ (n \in \mathbb{Z}) \ | \ ... \\
& v ::=  a \ | \ (v_1,...,v_k) \ | \ [v_1,...,v_l] \ | \ \{v_1,...,v_l\} 
\end{align*}
where $a$ is the atomic values or constants of types $\pi$, and $v$ are
 general values which can be a constant, a tuple with a small number ($k$) of components, a vertor or a stream with a large number ($l$) of elements.  

The expressions of SNESL are shown in Figure~\ref{fig-snesl-exps}.

\begin{figure}[h]
\begin{alignat*}{2}
& e &&::=  a \     \tag{constant} \\
&   && \quad | \ x  \tag{variable} \\
&   && \quad | \ (x_1,...,x_k) \tag{tuple}\\
&   && \quad | \ \Let{x}{e_1}{e_2} \tag{let-binding}\\
&   && \quad | \ \hcall{\Tupk{x}}  \tag{built-in function call} \\
&   && \quad | \ \Comp{e_1}{x}{y}{\usevars} \tag{general comprehension} \\
&   && \quad | \ \Recom{e_1}{x}{\usevars} \tag {restricted comprehension} 
\end{alignat*}
\caption{SNESL expressions \label{fig-snesl-exps}}
\end{figure}

As a minimalistic version of NESL, SNESL maintains the expressive and comfortable programming style of NESL. 
Basic expressions, such as the first five in Figure~\ref{fig-snesl-exps}, are almost the same as in NESL. 
The general comprehesion expression looks similar to the apply-to-each construct in NESL except an explicit list of the free variables (listed after the keyword $\*{using}$) occur in the body $e_1$. And the sieve is independent from apply-to-each  as the restricted comprehension for easier analysis and working as the only conditional expression in SNESL. 
Semantically, these comprehensions extend the apply-to-each to compute from vectors (i.e., type $[\varphi]$) to streams (i.e., type $\{\tau\}$). 

Another notable difference from NESL to SNESL occurs in the built-in functions. 

\begin{figure}[h]
\begin{alignat}{2} 
&\hcall && ::= \oplus \ | \ \ \*{append} \ | \ \*{concat} \ | \ \*{zip} \ | \ \*{iota}  \ | \ \*{part}  \ | \ \*{scan}_{\otimes} \ | \ \*{reduce}_{\otimes} \\
&   && \quad | \ \*{length} \ | \ \*{elt} \\
&   && \quad | \ \*{the}  \ | \ \*{empty} \\
&   && \quad | \ \*{mkseq} \ | \ \*{seq} \ | \ \*{tab} \\
& \oplus  \ && :: = + \ | \ - \ | \ \times \ |  \  \div \ | \ \% \ | \le \ | \ ... \tag{consts operations} \\
& \otimes \ && :: = + \ | \ \times  \ | \ ...  \tag{associative binary operations}
\end{alignat}
\caption{SNESL primitive functions \label{fig-snesl-func}}
\end{figure}

The functions listed in (1.1) and (1.2) of Figure~\ref{fig-snesl-func} are original supported in NESL, doing transformations on consts and vectors. In SNESL, list (1.1) are adapted to streaming versions with slight changes of parameter types where necessary.

Functions in (1.2) are kept their vector versions in SNESL to guarantee the efficiency, although it is possible to realize these functions in a streaming fashion.
List (1.3) are new primitives primarily working in  streams.  
The function $\*{the}$ can be used to a two if-else-then statement,  
$if e_0 then e_1 else e_2 $ 
$\*{empty}$, which tests if a collection is empty or not, is much more efficient in a streaming setting as it takes only constant complexity both in time and space.  
Finally, list (1.4) c convert between vectors and streams. $\*{seq}$, typed as $[\varphi] \rightarrow \{\varphi\}$, converts a vector to a sequence; $\*{tab}$ does the contrary work, tabulates the sequence into a vector.



\subsection{Cost model}
Based on the work-depth model, the cost model of SNESL is extended with a space
component, 


\section{Mathematical background and notations}
\begin{itemize}
	\item Set difference: \\
	For two sets $A$ and $B$, 
	\[ A \ \backslash \ B = \{s | s \in A \wedge s \notin B\} \]
	
	It is easy to prove the following properties: 
 	\begin{itemize}
	 	\item For any three sets $A,B$ and $C$: 
	 			\eq{set-p1}{(A \bs B) \cap C = (A \cap C) \bs B  =  A \cap (C \bs B)}
	 	\item For two sets $A$ and $B$,
	 	       \eq {set-p2}{A \cap B = \emptyset  \Leftrightarrow A \bs B = A}
 	\end{itemize}

\end{itemize}